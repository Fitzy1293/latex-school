%QUESTIONS 1, 3, 8, 16, 24, 29, 31, 39, 49 a) only, and 50
%Start at 79
%ANSWERS TO CHECK START ON PAGE 304

\documentclass[a4paper, 12pt]{article}
\usepackage[english]{babel}
%\usepackage[utf8]{inputenc}
\usepackage{fancyhdr}

%Reference pics from images subfolder.
\usepackage{graphicx}
\graphicspath{ {./images/} } %dir of embedded images, less clutter in main dir.

\usepackage{lipsum}
\usepackage[margin=20mm]{geometry} %Set margins
\usepackage{setspace}
\renewcommand{\baselinestretch}{1} %Change between pacing between elements.

\usepackage{amsmath}

\usepackage{hyperref}
\hypersetup{
    colorlinks=true,
    linkcolor=blue,
    filecolor=magenta,
    urlcolor=cyan,
}

\urlstyle{same}


\numberwithin{equation}{section}

                                                  %────────────────┐
                                                  \begin{document}%│
                                                  %────────────────┘
%────────────────────────────────────────────────────────────────────────────────────────────────────────────────────────────────
%Page header and footer
\pagestyle{fancy}
\fancyhf{}
\rhead{Owen Fitzgerald - Fall 2020}
\lhead{Class}
\rfoot{Page \thepage}
%────────────────────────────────────────────────────────────────────────────────────────────────────────────────────────────────
%Homework info
\begin{center}
  {\LARGE\textbf{Homework Template} \\
  {\large\emph{\textbf{Chapter x}}}
\end{center}

\begin{flushleft}

  {\textbf{Student:} \emph{\texttt{Owen Fitzgerald}} \\
  {\textbf{Professor:} \emph{\texttt{Dr. Ward}} \\
  \LaTeX\textbf{ File name:} \emph{\texttt{HW\_1\_Error\_Analysis\_Owen\_Fitzgerald.tex}} \\
  \footnotesize\url{https://github.com/Fitzy1293/latex-school/blob/main/HW_1_Error_Analysis_Owen_Fitzgerald.tex} \\

  \noindent\makebox[\linewidth]{\rule{\linewidth}{.4pt}}

\end{flushleft}
%────────────────────────────────────────────────────────────────────────────────────────────────────────────────────────────────


\noindent\makebox[\linewidth]{\rule{\linewidth}{1pt}}
%────────────────────────────────────────────────────────────────────────────────────────────────────────────────────────────────
\section{Problem 1}
\emph{To measure the activity of a radioactive sample, two students count the
alpha particles it emits. Student A watches for 3 minutes and counts 28 particles;
Student B watches for 30 minutes and counts 310 particles. (a) What should Student
A report for the average number emitted in 3 minutes, with his uncertainty? (b)
What should Student B report for the average number emitted in 30 minutes, with
her uncertainty? (c) What are the fractional uncertainties in the two measurements?
Comment.}\\

\subsection*{\emph{Solution 3.1\\Section 3.2: The Square-Root Rule for a Counting Experiment}}



\begin{flalign}


\begin{equation}
  student_{A_{ time}}= 3 \text{ minutes} \quad student_{A_{Emitted_{3 minutes}}} = 28 \text{ particles}\\
\end{equation}

\begin{equation}
  student_{B_{ time}}= 30 \text{ minutes} \quad student_{B_{Emitted_{3 minutes}}} =  310 \text{ particles}\\
\end{equation}


\end{flalign}

%======================================================================================================================================================

\begin{flushleft}
  \emph{(a) Student A's measurement}
  $$$
  \textrm{Using eq. 3.2 from the textbook:}  $\emph{Avg. events measurement} = \nu \pm{\sqrt{\nu}}$
  \textrm{ where} $\nu$ \textrm{- the greek letter \emph{nu} - is the best average.}

  $$$
\end{flushleft}


\begin{equation}
  student_{A_{Emitted_{3 minutes}}} = 28 \text{ particles}\\
\end{equation}

\begin{equation}
  student_{A_{ \text{ uncertainty} }} = \sqrt{28} \approx{5.29150262212918}   \\
\end{equation}

\begin{equation}
  student_{A_{ \text{ uncertainty} }} = \pm{5 \text{ particles}} \\
\end{equation}

\begin{equation}
  \boxed { student_{A_{Emitted_{3 minutes}}} = 28 \pm{5 \text{ particles}} \\ }
\end{equation}
%======================================================================================================================================================
\emph{(b) Student B's measurement}

\begin{equation}
  student_{B_{Emitted_{30 minutes}}} = 310 \text{ particles}\\
\end{equation}

\begin{equation}
  student_{B_{ \text{ uncertainty} }} = \sqrt{310} \approx{17.6068168616590}   \\
\end{equation}

\begin{equation}
  student_{B_{ \text{ uncertainty} }} = \pm{18 \text{ particles}} \\
\end{equation}

\begin{equation}
  \boxed { student_{B_{Emitted_{30 minutes}}} = 310 \pm{ \text{ 18 particles}} && \\ }
\end{equation}
\end{flalign}
%======================================================================================================================================================
\begin{flushleft}
  \emph{(c) What are the fractional uncertainties? Comment, i.e interpret the fractional uncertainties}
  $$$\textrm{Using eq .2.21 from the textbook:}  && $\text{fractional uncertainty} = \frac {\delta_\alpha} {|{x_{best}}|}$ $$$
\end{flushleft}


\noindent\makebox[\linewidth]{\rule{\linewidth}{1pt}}
\end{document}
