%QUESTIONS 1, 3, 8, 16, 24, 29, 31, 39, 49 a) only, and 50
%Start at 79
%ANSWERS TO CHECK START ON PAGE 304

\documentclass[a4paper, 12pt]{article}
\usepackage[english]{babel}
%\usepackage[utf8]{inputenc}
\usepackage{fancyhdr}

%Reference pics from images subfolder.
\usepackage{graphicx}
\graphicspath{ {./images/} } %dir of embedded images, less clutter in main dir.

\usepackage{lipsum}
\usepackage[margin=20mm]{geometry} %Set margins
\usepackage{setspace}
\renewcommand{\baselinestretch}{1} %Change between pacing between elements.

\usepackage{amsmath}

\usepackage{hyperref}
\hypersetup{
    colorlinks=true,
    linkcolor=blue,
    filecolor=magenta,
    urlcolor=cyan,
}

\urlstyle{same}


\numberwithin{equation}{section}


                                                  %────────────────┐
                                                  \begin{document}%│
                                                  %────────────────┘
%────────────────────────────────────────────────────────────────────────────────────────────────────────────────────────────────
%Page header and footer
\pagestyle{fancy}
\fancyhf{}
\rhead{Owen Fitzgerald - Fall 2020}
\lhead{Advanced Lab}
\rfoot{Page \thepage}
%────────────────────────────────────────────────────────────────────────────────────────────────────────────────────────────────
%Homework info
\begin{center}
  {\LARGE\textbf{Homework 1 - Error Analysis} \\
  {\large\emph{\textbf{Chapter 3}}}
\end{center}

\begin{flushleft}

  {\textbf{Student:} \emph{\texttt{Owen Fitzgerald}} \\
  {\textbf{Professor:} \emph{\texttt{Dr. Ward}} \\
  \LaTeX\textbf{ File name:} \emph{\texttt{HW\_1\_Error\_Analysis\_Owen\_Fitzgerald.tex}} \\
  \footnotesize\url{https://github.com/Fitzy1293/latex-school/blob/main/HW1.tex} \\
  \footnotesize\url{  https://github.com/Fitzy1293/latex-school/blob/main/HW1.pdf} \\


  \noindent\makebox[\linewidth]{\rule{\linewidth}{.4pt}}

\end{flushleft}
%────────────────────────────────────────────────────────────────────────────────────────────────────────────────────────────────


\noindent\makebox[\linewidth]{\rule{\linewidth}{1pt}}
%────────────────────────────────────────────────────────────────────────────────────────────────────────────────────────────────
\section{Problem 1}
\emph{To measure the activity of a radioactive sample, two students count the
alpha particles it emits. Student A watches for 3 minutes and counts 28 particles;
Student B watches for 30 minutes and counts 310 particles. (a) What should Student
A report for the average number emitted in 3 minutes, with his uncertainty? (b)
What should Student B report for the average number emitted in 30 minutes, with
her uncertainty? (c) What are the fractional uncertainties in the two measurements?
Comment.}\\

\subsection*{\emph{Solution 3.1\\Section 3.2: The Square-Root Rule for a Counting Experiment}}

  \begin{equation}
    student_{A\:Time}= 3 \text{ minutes} \quad student_{A\:emitted\:3\:minutes} = 28 \text{ particles}\\
  \end{equation}
  \begin{equation}
    student_{B\:Time}= 30 \text{ minutes} \quad student_{B\:emitted\:30\:minutes} =  310 \text{ particles}\\
  \end{equation}

%======================================================================================================================================================

\begin{flushleft}
  \emph{(a) Student A's measurement}

  Using eq. 3.2 from the textbook:
  $\emph{Avg. events measurement} = \nu \pm{\sqrt{\nu}}$
  where
  $\nu$
  - the greek letter \emph{nu} - is the best average.
\end{flushleft}


\begin{equation}
  student_{A\:emitted\:3\:minutes} =  28 \text{ particles}\\
\end{equation}

\begin{equation}
  student_{A\:uncertainty} = \sqrt{28} \approx{5.29150262212918}   \\
\end{equation}

\begin{equation}
  student_{A\:uncertainty} = \pm{5 \text{ particles}} \\
\end{equation}

\begin{equation}
  \boxed { student_{A\:emitted\:3\:minutes} =  28 \pm{5 \text{ particles}} }
\end{equation}
%======================================================================================================================================================
\emph{(b) Student B's measurement}
\begin{equation}
  student_{B\:emitted\:30\:minutes} = 310 \text{ particles}\\
\end{equation}

\begin{equation}
  student_{B_{\:uncertainty} } = \sqrt{310} \approx{17.6068168616590}   \\
\end{equation}

\begin{equation}
  student_{B_{\:uncertainty} } = \pm{18 \text{ particles}} \\
\end{equation}

\begin{equation}
  \boxed { student_{B\:emitted\:30\:minutes} = 310 \pm{18\:particles} \\ }
\end{equation}

%======================================================================================================================================================
%\begin{flushleft}
  \emph{(c) What are the fractional uncertainties? Comment, i.e interpret, the fractional uncertainties.}\\

  Using eq .2.21 from the textbook:
  $\emph{fractional uncertainty} = \frac {\delta_x} {|{x_{best}}|}$
  , we can use eq. (1.11) with Students A and B.

  \begin{center}
  \begin{equation}
    \emph{fractional uncertainty} = \frac {\delta_\nu} {|{\nu_{best}}|} = \frac { \sqrt {student_{\:particle\:count\:best} }} {student_{\:particle\:count\:best}}
  \end{equation}
  \end{center}
%\end{flushleft}

\begin{equation}
  student_A_{\:fractional\:uncertainty} = \frac {5} {28} \approx{0.178571428571429}
\end{equation}
\begin{equation}
  student_B_{\:fractional\:uncertainty} = \frac {18} {310} \approx{0.0580645161290323}
\end{equation}

\begin{subequations}
\begin{empheq}[box=\widefbox]{align}
  student_A_{\:fractional\:uncertainty} = 18\% \\
  student_B_{\:fractional\:uncertainty} = 6\%
\end{empheq}
\end{subequations}

Student B's total uncertainty is higher than student A's, however B has a lower fractional uncertainty.
Counting more events will always reduce your fractional uncertainty in these cases.
A lower fractional uncertainty implies a more accurate measurement, so B has a better measurement.

\noindent\makebox[\linewidth]{\rule{\linewidth}{1pt}}
%======================================================================================================================================================

\section{Problem 2}
\emph{Most of the ideas of error analysis have important applications in many
different fields. This applicability is especially true for the square-root rule (3.2)
for counting experiments, as the following example illustrates. The normal average
incidence of a certain kind of cancer has been established as 2 cases per 10,000
people per year. The suspicion has been aired that a certain town (population
20,000) suffers a high incidence of this cancer because of a nearby chemical dump.
To test this claim, a reporter investigates the town's records for the past 4 years and
finds 20 cases of the cancer. He calculates that the expected number is 16 (check
this) and concludes that the observed rate is 25% more than expected. Is he justified
in claiming that this result proves that the town has a higher than normal rate for
this cancer?}\\

\subsection*{\emph{Solution 3.3\\Section 3.2: Abnormal number of cancer cases?}}

\begin{equation}
  town_{expected\:cases}= 16 \quad town_{actual\:cases} = 20
\end{equation}


\begin{equation}
\sqrt{town_{actual\:cases} } = \sqrt{20} \approx{4.47213595499958}
\end{equation}
\begin{equation}
\sqrt{town_{actual\:cases} } = \pm{4} \text{ cases}
\end{equation}
The fewest possible cases are seen by substracting the uncertainty $20 - 4 = 16$.
This is within the expected range, so he is not justified in his claim.

\noindent\makebox[\linewidth]{\rule{\linewidth}{1pt}}
%======================================================================================================================================================
\section{Problem 3}
\emph{Binomial theorem exploration.}
\subsection*{\emph{Solution 3.8\\Section 3.3: Explore how the binomial theorem works.}}

\begin{equation}
\left( 1 + x\right)^n
\end{equation}


\end{document}
