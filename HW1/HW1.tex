%QUESTIONS 1, 3, 8, 16, 24, 29, 31, 39, 49 a) only, and 50
%Start at 79
%ANSWERS TO CHECK START ON PAGE 304

\documentclass[a4paper, 12pt]{article}
\usepackage[english]{babel}
%\usepackage[utf8]{inputenc}
\usepackage{fancyhdr}

%Reference pics from images subfolder.
\usepackage{graphicx}
\graphicspath{ {./images/} } %dir of embedded images, less clutter in main dir.

\usepackage{lipsum}
\usepackage[margin=20mm]{geometry} %Set margins
\usepackage{setspace}
\renewcommand{\baselinestretch}{1} %Change between pacing between elements.

\usepackage{amsmath}

\usepackage{hyperref}
\hypersetup{
    colorlinks=true,
    linkcolor=blue,
    filecolor=magenta,
    urlcolor=cyan,
}

\urlstyle{same}


\numberwithin{equation}{section}
\usepackage{empheq}
\newcommand*\widefbox[1]{\fbox{\hspace{2em}#1\hspace{2em}}}

\usepackage{parskip}

                                                  %────────────────┐
                                                  \begin{document}%│
                                                  %────────────────┘

%────────────────────────────────────────────────────────────────────────────────────────────────────────────────────────────────
%Page header and footer
\pagestyle{fancy}
\fancyhf{}
\rhead{Owen Fitzgerald - Fall 2020}
\lhead{Advanced Lab}
\rfoot{Page \thepage}
%────────────────────────────────────────────────────────────────────────────────────────────────────────────────────────────────
%Homework info
\begin{center}
  {\LARGE\textbf{Homework 1 - Error Analysis} \\
  {\large\emph{\textbf{Chapter 3}}}
\end{center}

\begin{flushleft}

  {\textbf{Student:} \emph{\texttt{Owen Fitzgerald}} \\
  {\textbf{Professor:} \emph{\texttt{Dr. Ward}} \\
  \LaTeX\textbf{ File name:} \emph{HW1.tex}} \\
  \footnotesize\url{https://github.com/Fitzy1293/latex-school/blob/main/HW1/HW1.tex} \\
  \footnotesize\url{https://github.com/Fitzy1293/latex-school/blob/main/HW1/HW1.pdf} \\


  \noindent\makebox[\linewidth]{\rule{\linewidth}{.4pt}}

\end{flushleft}
%────────────────────────────────────────────────────────────────────────────────────────────────────────────────────────────────


\noindent\makebox[\linewidth]{\rule{\linewidth}{1pt}}
%────────────────────────────────────────────────────────────────────────────────────────────────────────────────────────────────
\section{Problem 1}
\emph{To measure the activity of a radioactive sample, two students count the
alpha particles it emits. Student A watches for 3 minutes and counts 28 particles;
Student B watches for 30 minutes and counts 310 particles. (a) What should Student
A report for the average number emitted in 3 minutes, with his uncertainty? (b)
What should Student B report for the average number emitted in 30 minutes, with
her uncertainty? (c) What are the fractional uncertainties in the two measurements?
Comment.}\\

\subsection*{\emph{Solution 3.1\\Section 3.2: The Square-Root Rule for a Counting Experiment}}

  \begin{equation}
    student_{A\:Time}= 3 \text{ minutes} \quad student_{A\:emitted\:3\:minutes} = 28 \text{ particles}\\
  \end{equation}
  \begin{equation}
    student_{B\:Time}= 30 \text{ minutes} \quad student_{B\:emitted\:30\:minutes} =  310 \text{ particles}\\
  \end{equation}

%======================================================================================================================================================

\begin{flushleft}
  \emph{(a) Student A's measurement}

  Using eq. 3.2 from the textbook:
  $\emph{Avg. events measurement} = \nu \pm{\sqrt{\nu}}$
  where
  $\nu$
  - the greek letter \emph{nu} - is the best average.
\end{flushleft}


\begin{equation}
  student_{A\:emitted\:3\:minutes} =  28 \text{ particles}\\
\end{equation}

\begin{equation}
  student_{A\:uncertainty} = \sqrt{28} \approx{5.29150262212918}   \\
\end{equation}

\begin{equation}
  student_{A\:uncertainty} = \pm{5 \text{ particles}} \\
\end{equation}

\begin{equation}
  \boxed { student_{A\:emitted\:3\:minutes} =  28 \pm{5 \text{ particles}} }
\end{equation}
%======================================================================================================================================================
\emph{(b) Student B's measurement}
\begin{equation}
  student_{B\:emitted\:30\:minutes} = 310 \text{ particles}\\
\end{equation}

\begin{equation}
  student_{B_{\:uncertainty} } = \sqrt{310} \approx{17.6068168616590}   \\
\end{equation}

\begin{equation}
  student_{B_{\:uncertainty} } = \pm{18 \text{ particles}} \\
\end{equation}

\begin{equation}
  \boxed { student_{B\:emitted\:30\:minutes} = 310 \pm{18\:particles} \\ }
\end{equation}

%======================================================================================================================================================
%\begin{flushleft}
  \emph{(c) What are the fractional uncertainties? Comment, i.e interpret, the fractional uncertainties.}\\

  Using eq .2.21 from the textbook:
  $\emph{fractional uncertainty} = \frac {\delta_x} {|{x_{best}}|}$
  , we can use eq. (1.11) with Students A and B.

  \begin{center}
  \begin{equation}
    \emph{fractional uncertainty} = \frac {\delta_\nu} {|{\nu_{best}}|} = \frac { \sqrt {student_{\:particle\:count\:best} }} {student_{\:particle\:count\:best}}
  \end{equation}
  \end{center}
%\end{flushleft}

\begin{equation}
  student_A_{\:fractional\:uncertainty} = \frac {5} {28} \approx{0.178571428571429}
\end{equation}
\begin{equation}
  student_B_{\:fractional\:uncertainty} = \frac {18} {310} \approx{0.0580645161290323}
\end{equation}

\begin{subequations}
\begin{empheq}[box=\widefbox]{align}
  student_A_{\:fractional\:uncertainty} = 18\% \\
  student_B_{\:fractional\:uncertainty} = 6\%
\end{empheq}
\end{subequations}

Student B's total uncertainty is higher than student A's, however B has a lower fractional uncertainty.
Counting more events will always reduce your fractional uncertainty in these cases.
A lower fractional uncertainty implies a more accurate measurement, so B has a better measurement.

\noindent\makebox[\linewidth]{\rule{\linewidth}{1pt}}
%────────────────────────────────────────────────────────────────────────────────────────────────────────────────────────────────
\section{Problem 2}
\emph{Most of the ideas of error analysis have important applications in many
different fields. This applicability is especially true for the square-root rule (3.2)
for counting experiments, as the following example illustrates. The normal average
incidence of a certain kind of cancer has been established as 2 cases per 10,000
people per year. The suspicion has been aired that a certain town (population
20,000) suffers a high incidence of this cancer because of a nearby chemical dump.
To test this claim, a reporter investigates the town's records for the past 4 years and
finds 20 cases of the cancer. He calculates that the expected number is 16 (check
this) and concludes that the observed rate is 25% more than expected. Is he justified
in claiming that this result proves that the town has a higher than normal rate for
this cancer?}\\

\subsection*{\emph{Solution 3.3\\Section 3.2: The Square-Root Rule for a Counting Experiment}}

\begin{equation}
  town_{expected\:cases}= 16 \quad town_{actual\:cases} = 20
\end{equation}


\begin{equation}
\sqrt{town_{actual\:cases} } = \sqrt{20} \approx{4.47213595499958}
\end{equation}
\begin{equation}
\sqrt{town_{actual\:cases} } = \pm{4} \text{ cases}
\end{equation}

The fewest possible cases are seen by substracting the uncertainty
$20 - 4 = 16$.
This is within the expected range, so he is not justified in his claim.

\noindent\makebox[\linewidth]{\rule{\linewidth}{1pt}}
%────────────────────────────────────────────────────────────────────────────────────────────────────────────────────────────────
\section{Problem 3}
\emph{Binomial theorem exploration.}
\subsection*{\emph{Solution 3.8\\Section 3.3: Sums and Differences; Products and Quotients}}

\begin{equation}
  ( 1 + x)^n = 1 + nx + \frac{n( n - 1  )} {1 \cdot{ 2}} x^2 + \frac{n( n - 1  ) ( n - 2)} {1 \cdot{2} \cdot{3}} x^3 + ...
\end{equation}

\emph{(a)}

$n=2$
case
\begin{equation}
( 1 + x)^2 = 1 + 2x + \frac{2( 2 - 1  )} {1 \cdot{ 2}} x^2 + \frac{2( 2 - 1  ) ( 2 - 2)} {1 \cdot{2} \cdot{3}} x^3 + ...
\end{equation}

All terms that include $(n-2)$ will cancel; RHS terms to a higher power of x than 2 will go to 0.

\begin{equation}
( 1 + x)^2 = 1 + 2x + \frac{2( 2 - 1  )} {1 \cdot{ 2}} x^2
\end{equation}

\begin{equation}
( 1 + x)^2 = x^2 + 2x + 1
\end{equation}

\begin{equation}
( 1 + x)^2 = (x + 1) (x+1)
\end{equation}

\begin{equation}
\boxed{( 1 + x)^2 = ( 1 + x)^2}
\end{equation}

$n=3$
case
\begin{equation}
    ( 1 + x)^n = 1 + nx + \frac{n( n - 1  )} {1 \cdot{ 2}} x^2 + \frac{n( n - 1  ) ( n - 2)} {1 \cdot{2} \cdot{3}} x^3 +  \frac{n(n-1)(n-2)(n-3)} {1\cdot{2}\cdot{3}\cdot{4}} x^4
    + ...
\end{equation}

All terms that include $(n-3)$ will cancel; RHS terms to a higher power of x than 3 will go to 0.

\begin{equation}
    ( 1 + x)^3 = 1 + 3x + \frac{3( 3 - 1  )} {1 \cdot{ 2}} x^2 + \frac{3( 3 - 1  ) ( 3 - 2)} {1 \cdot{2} \cdot{3}} x^3 +  \frac{3(3-1)(3-2)(3-3)} {1\cdot{2}\cdot{3}\cdot{4}} x^4
    + ...
\end{equation}

\begin{equation}
    ( 1 + x)^3 = 1 + 3x + 3 x^2 + x^3
\end{equation}

\begin{equation}
    \boxed{( 1 + x)^3 = ( 1 + x)^3}
\end{equation}

\emph{(b)}

$n=-1$
case
\begin{equation}
  ( 1 + x)^{-1} = 1 + -1x + \frac{-1( -1 - 1  )} {1 \cdot{ 2}} x^2 + \frac{-1( -1 - 1  ) ( -1 - 2)} {1 \cdot{2} \cdot{3}} x^3 + ...
\end{equation}

\begin{equation}
  \boxed{( 1 + x)^{-1} = 1 - x + x^2 - x^3 + ...}
\end{equation}

Now use
$( 1 + x)^{-1} \approx{1 - x}$
for
$x =0.5, 0.1, 0.01$
and find how much it differs from the exact value.

\begin{equation}
( 1 + .5)^{-1} \approx{1 - .5}
\end{equation}
\begin{equation}
.67 \approx{.5}
\end{equation}
\begin{equation}
dif. = .67 - .5 = .17
\end{equation}
\begin{equation}
\boxed { percent\:error = (.17 / .67) * 100\% = 25\% }
\end{equation}

\begin{equation}
( 1 + .1)^{-1} \approx{1 - .1}
\end{equation}
\begin{equation}
.91 \approx{.9}
\end{equation}
\begin{equation}
dif = .91 - .9 = .01
\end{equation}
\begin{equation}
\boxed { percent\:error. = (.01 / .91) * 100\% = 1\% }
\end{equation}

\begin{equation}
( 1 + .01)^{-1} \approx{1 - .01}
\end{equation}
\begin{equation}
.9901 \approx{.99}
\end{equation}
\begin{equation}
dif = .9901  - .99 = .0001
\end{equation}
\begin{equation}
\boxed { percent\:error. = (.0001 / .9901) * 100\% = .01\% }
\end{equation}
%────────────────────────────────────────────────────────────────────────────────────────────────────────────────────────────────
\section{Problem 4}
\emph{A student measures five lengths:
a = 50 ± 5,
b
= 30 ± 3, c = 60 ± 2,
d
= 40 ± 1, e = 5.8 ± 0.3
(all in cm) and calculates the four sums a + b, a + c, a + d, a + e. Assuming the
original errors were independent and random, find the uncertainties in her four an-
swers [rule (3.13), "errors add in quadrature"]. If she has reason to think the original
errors were not independent, what would she have to give for her final uncertainties
[rule (3.14), "errors add directly"]? Assuming the uncertainties are needed with only
one significant figure, identify those cases in which the second uncertainty (that in
b, c, d, e) can be entirely ignored. If you decide to do the additions in quadrature
on a calculator, note that the conversion from rectangular to polar coordinates auto-matically calculates
$\sqrt{x^2 + y^2}$ for given x and y.}
\subsection*{\emph{Solution 3.16\\Section 3.5: Independent Uncertainties in a Sum}}\\

Equation 3.13 in the textbook
$\delta_q = \sqrt{ {\delta_x}^2 +  {\delta_y}^2 }  $
gives uncertainty for measurements that have independent and random error.

Equation 3.14 in the textbook
$\delta_q \approx{\delta_x +  \delta_y } $
gives a larger uncertainty for when 3.13's conditions aren't met.


\begin{equation}
 \delta_a_b_{\:quad.} = \sqrt{ {\delta_a}^2 +  {\delta_b}^2 } = \sqrt{5^2 + 3^2} = \sqrt{36}
\end{equation}
\begin{equation}
 \delta_a_b_{\:sum} \approx{\delta_a +  \delta_b = 5 + 3}
\end{equation}
\begin{subequations}
\begin{empheq}[box=\widefbox]{align}
  \delta_a_b_{\:best}= 80\:cm\\
   \delta_a_b_{\:quad.} = \pm{6}\:cm\\
   \delta_a_b_{\:sum} = \pm{8}\:cm
\end{empheq}
\end{subequations}



\begin{equation}
 \delta_a_c_{\:quad.} = \sqrt{ {\delta_a}^2 +  {\delta_c}^2 } = \sqrt{5^2 + 2^2} = \sqrt{29}
\end{equation}
\begin{equation}
 \delta_a_c_{\:sum} \approx{\delta_a +  \delta_c = 5 + 2}
\end{equation}
\begin{subequations}
\begin{empheq}[box=\widefbox]{align}
  \delta_a_b_{\:best}= 110\:cm\\
   \delta_a_c_{\:quad.} = \pm{5}\:cm\\
   \delta_a_c_{\:sum} = \pm{7}\:cm
\end{empheq}
\end{subequations}

\begin{equation}
 \delta_a_d_{\:quad.} = \sqrt{ {\delta_a}^2 +  {\delta_d}^2 } = \sqrt{5^2 + 1^2} = \sqrt{26}
\end{equation}
\begin{equation}
 \delta_a_d_{\:sum} \approx{\delta_a +  \delta_d = 5 + 1}
\end{equation}
\begin{subequations}
\begin{empheq}[box=\widefbox]{align}
  \delta_a_b_{\:best}= 90\:cm\\
   \delta_a_d_{\:quad.} = \pm{5}\:cm\\
   \delta_a_d_{\:sum} = \pm{6}\:cm
\end{empheq}
\end{subequations}



\begin{equation}
 \delta_a_e_{\:quad.} = \sqrt{ {\delta_a}^2 +  {\delta_e}^2 } = \sqrt{5^2 + 0.3^2} = \sqrt{25.09}
\end{equation}
\begin{equation}
 \delta_a_e_{\:sum} \approx{\delta_a +  \delta_e = 5 + 0.3}
\end{equation}
\begin{subequations}
\begin{empheq}[box=\widefbox]{align}
  \delta_a_b_{\:best}= 55.8\:cm\\
   \delta_a_e_{\:quad.} = \pm{5}\:cm\\
   \delta_a_e_{\:sum} = \pm{5.3}\:cm
\end{empheq}
\end{subequations}

There is one equation where you cannot ignore the quadratic uncertainty, $\delta_q = a + b$.
Both the quadratic and sum uncertainties round to the same thing, for both the positive and negative cases respectively.\\
The other measurements end up rounding to different things, depending on which uncertainty you use.
The range of possible measurements differs between the two uncertainties, so the straight sum needs to be used.
%────────────────────────────────────────────────────────────────────────────────────────────────────────────────────────────────
\section{Problem 5}
\emph{Charge to mass ratio independet uncertainties}

\subsection*{\emph{Solution 3.24\\Section 3.6: More About Independent Uncertainties}}\\

\emph{(a) Solving for r, electron charge to mass ratio. Include uncertainty.}\\

\begin{equation}
r = \frac{125}{32 \mu^2_o N^2} \frac{D^2 V}{d^2 I^2}
\end{equation}

\begin{equation}
r_{best} = \frac{125}{32 (4 \pi \times 10^{-7} kg \cdot m \cdot s^{-2} \cdot A^{-2}) ^2 (72)^2} \frac{(0.661 \:m)^2 (45.0 \:V)}{(0.0914 \:m)^2 (2.48 \:A)^2}
\end{equation}
Changed $\mu_0$ 's from newtons to avoid confusion with N, the coil number.

\begin{equation}
  r_{best} = 1.83 \times 10^{11} C/kg
\end{equation}


Soolve with  equation 3.18 from the textbook, which explains how to get uncertainties in functions with multiple variables in products and quotients.

\begin{equation}
\frac {\delta_r}{r_{best}} =
  \sqrt{
    {\left( 2 \frac{\delta D}{D} } \right)}^2 +
    {\left(\frac{\delta V}{V} } \right)}^2 +
    {\left( 2 \frac{\delta d}{d} } \right)}^2 +
    {\left( 2 \frac{\delta I}{I} } \right)}^2
    }
\end{equation}

\begin{equation}
\delta_r =
  r_{best} \cdot \sqrt{
    {\left( 2 \frac{\delta D}{D} } \right)}^2 +
    {\left(\frac{\delta V}{V} } \right)}^2 +
    {\left( 2 \frac{\delta d}{d} } \right)}^2 +
    {\left( 2 \frac{\delta I}{I} } \right)}^2
    }
\end{equation}

\begin{equation}
\delta_r =
  1.83 \times 10^{11} \cdot
  \sqrt{
    {\left( 2 \frac{0.002}{0.661} } \right)}^2 +
    {\left(\frac{0.2}{45.0} } \right)}^2 +
    {\left( 2 \frac{0.0005}{0.0914} } \right)}^2 +
    {\left( 2 \frac{0.04}{2.48} } \right)}^2
  }
  C / kg
\end{equation}

\begin{equation}
\delta_r \approx{\pm 6.38 \times 10 ^ 9} C / kg
\end{equation}

\begin{equation}
\delta_r = \pm 0.06 \times 10 ^{11} C / kg
\end{equation}

\begin{equation}
\boxed { r = 1.83 \times 10^{11} \pm 0.06 \times 10 ^{11} C / kg }
\end{equation}

\emph{(b) Compare measured value to accepted value.}

\begin{equation}
percent \: error = \frac{ (1.83 - 1.759) \times 10^{11} }{1.759 \times 10^{11}} * 100\%
\end{equation}

\begin{equation}
\boxed {percent \: error = 4\%}
\end{equation}

This value is in the acceptable range for percent error, so this was a valid recreation of the experiment.


\end{document}

\begin{equation}

\end{equation}




%QUESTIONS 1, 3, 8, 16, 24, 29, 31, 39, 49 a) only, and 50
%Start at 79
%ANSWERS TO CHECK START ON PAGE 304
