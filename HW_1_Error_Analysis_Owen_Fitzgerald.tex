\documentclass[a4paper, 12pt]{article}
\usepackage[english]{babel}
%\usepackage[utf8]{inputenc}
\usepackage{fancyhdr}

%Reference pics from images subfolder.
\usepackage{graphicx}
\graphicspath{ {./images/} } %dir of embedded images, less clutter in main dir.

\usepackage{lipsum}
\usepackage[margin=20mm]{geometry} %Set margins
\usepackage{setspace}
\renewcommand{\baselinestretch}{1} %Change between pacing between elements.

\usepackage{amsmath}
\numberwithin{equation}{section}
%───────────────────────────────────────────────────────────────────────────────────────────────────────────────────────────────

                                                  %────────────────┐
                                                  \begin{document}%│
                                                  %────────────────┘
%────────────────────────────────────────────────────────────────────────────────────────────────────────────────────────────────
%Page header and footer
\pagestyle{fancy}
\fancyhf{}
\rhead{Owen Fitzgerald - Fall 2020}
\lhead{Class}
\rfoot{Page \thepage}
%────────────────────────────────────────────────────────────────────────────────────────────────────────────────────────────────
%Homework info
\begin{center}
  {\LARGE\textbf{Homework Template} \\
  {\large\emph{\textbf{Chapter x}}}
\end{center}

\begin{flushleft}

  {\textbf{Student:} \emph{\texttt{Owen Fitzgerald}} \\
  {\textbf{Professor:} \emph{\texttt{Dr. Ward}} \\
  \LaTeX\textbf{ Path:} \emph{\texttt{HW\_1\_Error\_Analysis\_Owen\_Fitzgerald.tex}} \\

  \noindent\makebox[\linewidth]{\rule{\linewidth}{.4pt}}

\end{flushleft}
%────────────────────────────────────────────────────────────────────────────────────────────────────────────────────────────────

%QUESTIONS 1, 3, 8, 16, 24, 29, 31, 39, 49 a) only, and 50
\noindent\makebox[\linewidth]{\rule{\linewidth}{1pt}}
%────────────────────────────────────────────────────────────────────────────────────────────────────────────────────────────────
\section{Problem 1}
\emph{To measure the activity of a radioactive sample, two students count the
alpha particles it emits. Student A watches for 3 minutes and counts 28 particles;
Student B watches for 30 minutes and counts 310 particles. (a) What should Student
A report for the average number emitted in 3 minutes, with his uncertainty? (b)
What should Student B report for the average number emitted in 30 minutes, with
her uncertainty? (c) What are the fractional uncertainties in the two measurements?
Comment.}\\

\subsection*{\emph{Solution 3.1\\Section 3.2: The Square-Root Rule for a Counting Experiment}}



\begin{flalign}


\begin{equation}
  student_{A_{ time}}= 3 \text{ minutes} = 180 \cdot s \quad student_{A_{ \text{ emission rate} }} =  \frac{28}{3} \cdot \frac{particles}{minute}\\
\end{equation}

\begin{equation}
  student_{B_{ time}}= 3 \text{ minutes} = 180 \cdot s \quad student_{B_{ \text{ emission rate} }} =  \frac{300}{3} \cdot \frac{particles}{minute}\\
\end{equation}

\end{flalign}

\emph{(a)}

\begin{equation}
  student_{A_{ \text{ Emitted in 3 minutes} }} = 28\\
\end{equation}

\begin{equation}
  student_{A_{ \text{ uncertainty} }} = \sqrt{28} \approx{5.29150262212918...}   \\
\end{equation}
\begin{equation}
  student_{A_{ \text{ uncertainty} }} = \pm{5 particles emmited} \\
\end{equation}
\end{flalign}


\noindent\makebox[\linewidth]{\rule{\linewidth}{1pt}}



\end{document}
